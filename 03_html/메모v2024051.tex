<시작태그 속성+"값"속성="값"...>내용</끝태그>
<html lang+"ko">내용</html>
<head>내용</head>
웹표준,웹접근성(컴터가 있을때 장애인이고 일반인이고 접근성에 가능해야 한다.),비표준
<h1>제목</h1>
<h2>제목</h2>
<h3>제목</h3>
<P> 나는 문단이다.</p>
<div> 영역 분할</div>
<section>sksms tprtus</section>
<a>링크</a>
<img>
<span> 인라인 요소 스타일 지정 </span>

<body>내용</body>
<meta charset="UTF-8"> 
<title>회사명|블랜드명</title>       |=또는 이라는 의미
<h1> 제목<h1>       <H1>큰제목
<h2>콘텐츠제목</H2>
<img> ->이미지 가지고 오기 /이미지 이다. 뒷쪽 헤그테그 마무리 </> 이런것이 없다 .
<img src="값" alt="값">  장애인 일경우 소리로   이미지 옆에 
예-<img src="피그마.png" alt="피그마 로고">
<h3>콘텐츠 제목</H3>     
<P> 나는 단락이다.</P>
<a>링크</a>
<a href(필수 속성) ="값">커피</a>  
글로벌 속성 <h1 class="값">  class  글로벌 속성 어디에나 쓴다 하지만 꼭 안써도 된다.
bady 에만 쓴다 헤드와 메타 타이틀 안에 쓰면 안된다
속성은 ,글로벌 속성,필수 속성, 선택 속성이 있다.

속성은 태그마다 정해져 있다. 
퀴즈>HTML 내용은 어떤걸 써야하나?1. 태그   (내용은 꼭 태그를 써야한다 ) 2.미디어(텍스트,비디오,오디오)
내용= 태그를 의미한다.주의 -끝태그가 없는 태그가 있다.->빈요소(element) 퀴즈>HTM내용은 어떤걸 써야하나? 태그(tag)MATA는 종료 태그가 없다 . 내용이 필요가 없기때문에 종료 태그가 필요 없다 내용이 있는 테그는 종료 태그가 있다.
<body>내용</body> 태그는 앞에 가로안 뒤로 가로안을 이야기 한다 /요소(element)는 내용을 말한다 
<tite>네이버</tite>


</body>

속성은 태그마다 정해져 있다.
